%---------------------------------------------------------------------------%
%->> Frontmatter
%---------------------------------------------------------------------------%
%-
%-> 生成封面
%-
\maketitle% 生成中文封面
\MAKETITLE% 生成英文封面
%-
%-> 作者声明
%-
\makedeclaration% 生成声明页
%-
%-> 中文摘要
%-
\intobmk\chapter*{摘\quad 要}% 显示在书签但不显示在目录
\setcounter{page}{1}% 开始页码
\pagenumbering{Roman}% 页码符号

得益于实验制备、操控、测量的便捷性和多样性,冷原子物理研究进展快速。目前的研究重点不仅在于冷原子体系自身特性的挖掘,更进一步其作为量子模拟的平台正受到越来越多的关注。固体物理中的众多概念相继被冷原子模拟平台实现,越来越多在固体物理中难以实现的模型、难以探索的区域在冷原子平台中得到探索。这些新的探索发现了越来越多的新奇物理特性,这些新奇特性的解释与发现相互促进,使得这一领域研究结果频出。

本文便沿着这样的思路,聚焦于最近冷原子模拟平台揭示的物理来展开理论研究。在本文的第一章中我们简要介绍了近期实验相关的背景物理介绍,主要包括:少体物理、自旋交换相互作用、费米极化子、热化动力学。第二章我们用数值对角化的方法研究了带有自旋交换相互作用的一维少体磁性杂质体系,发现两体与三体能谱揭示出反铁磁耦合下基态局域自旋的屏蔽现象。进一步我们找到了一系列铁磁支,其波函数已知且具有良好的自旋-电荷分离行为。借助目前实验进展,实验中可以制备并探测这一体系。在第三章,我们进入多体非磁性杂质——费米极化子的研究。通过统一的变分波函数我们系统地研究不同维度下极化子到分子的转变,验证这一转变在三维与二维下存在,在一维下不存在。揭示这一转变的本质在于基态从零动量到费米动量的转移,基于此发现了分子态的巨大简并。进一步采用局域密度近似讨论了有限温度有限密度下的单粒子实验可观测量的连续化问题,并与近期实验做了比较。在第四章我们从能谱静态性质的研究来到了非平衡热化动力学特性的研究,利用数值严格对角化讨论了带有纵场的横场伊辛模型从初态$|Z_2\rangle$出发的局域算符测量的热化相图,选取不同参数区间做局域可观测量的时间平均与热力学平均做对比来确定是否热化,我们发现了里德堡区域的非热化行为以及经典极限下基态附近的弱热化行为,并尝试建立不同区域间的联系。最后第五章,我们总结这一从少体到多体再到动力学的研究,并给出未来值得继续探索问题的思考。

\keywords{杂质,少体,自旋交换,极化子,热化}% 中文关键词


\intobmk\chapter*{Abstract}% 显示在书签但不显示在目录

Cold atom physics has made great progress thanks to the advantage on preparing, manipulating and measuring. Recent attention has not only been focused on intrinsic properties of cold atom systems, but also on quantum simulation. Many phenomena in solid state physics has been observed on cold atom platforms. More and more models and regions that cannot be explored experimentally in solid state physics have been successfully realized in cold atom. These new experimental progress will show more and more interesting results. Theoretical explanation for these progress and extensions follows quickly. Both of them stimulate each other generously, which makes the whole community flourishing.

This thesis follows along this logic, which focus on recent experimental progress to get down research. In Chapter I we generally introduce some background and methods relating to four hot systems, which mainly includes : few body system, spin-exchange system, Fermi polaron and thermalization system. In Chapter II we use exact diagonalization to obtain whole spectrum of two body and three body magnetic impurity system with spin exchange interaction in 1D. We find screening effect in anti-ferromagnetic coupling. We go on illustrating a series of ferromagnetic branches. They have exact wave function of spin and charge separation. This few body system can be realized directly and the special correlation coulde be detected. In chapter III we enter into non-magnetic many body impurity system: Fermi polaron. We use unified variational wave function with up to 2 p-h excitation to systematically study polaron-molecule transition in different dimensions. We confirm the existence of this transition in 3D and 2D. We show that nature of this transition lies on transformation of ground state momentum from 0 to Fermi momentum. Furthermore, we show huge degeneracy of molecule states. With the help of local density approximation we calculate trap-averaged single particle properties in realistic system with finite density and finite temperature. Comparison with experiment data is also shown. In chapter IV we consider non-equilibrium thermalization dynamics. By exact diagonalization we study thermalization phase diagram of TFIM with longitudinal field starting from $|Z_2\rangle$. Choosing different parameters, we take comparison between long time average and thermal Gibbs average as justification of thermalization. We found non-thermalization in Rydberg regime and weak thermalization nearby ground state, we try bridging different regimes. Finally, we summarize our research along this logic: from few to many to dynamics and show our thoughts on directions worthy further investigation in chapter V.

\KEYWORDS{Impurity, Few Body, Spin-exchange, Polaron, Thermalization}% 英文关键词
%---------------------------------------------------------------------------%
