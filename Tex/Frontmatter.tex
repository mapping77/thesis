%---------------------------------------------------------------------------%
%->> Frontmatter
%---------------------------------------------------------------------------%
%-
%-> 生成封面
%-
\maketitle% 生成中文封面
\MAKETITLE% 生成英文封面
%-
%-> 作者声明
%-
\makedeclaration% 生成声明页
%-
%-> 中文摘要
%-
\intobmk\chapter*{摘\quad 要}% 显示在书签但不显示在目录
\setcounter{page}{1}% 开始页码
\pagenumbering{Roman}% 页码符号

冷原子物理研究因实验制备、操控、测量的便捷和丰富而进展快速。目前的研究不仅在于冷原子实验体系自身特性的挖掘,而且可以作为量子模拟的实验平台来探索凝聚态物理中新奇概念。大量固体物理中的概念与现象被冷原子模拟平台所实现,越来越多在固体物理中难以实现的模型、难以探索的区域在冷原子平台中得到探索。这些不曾被实现的模型正表现出越来越多的新奇特性,对这些新奇特性的理论解释与这些新奇现象紧密结合,相互促进。

本文便沿着这样的思路,聚焦于最近冷原子实验平台所探索到的新区域来展开理论研究。在本文的第一章中我们简要介绍了基于近期实验进展的相关背景物理介绍,主要包括:少体物理、自旋交换相互作用、费米极化子、热化动力学。第二章我们用数值对角化的方法研究了带有自旋交换相互作用的一维少体磁性杂质体系,研究两体与三体能谱揭示出反铁磁耦合下基态局域自旋的“屏蔽”现象。进一步我们发现了一系列upper branch,其中以铁磁branch最为特殊,其波函数已知且具有良好的自旋-电荷分离行为。借助目前的实验进展这一体系是可以被实验所制备的,其中的特殊关联可直接验证。在第三章,我们进入多体非磁性杂质——费米极化子的研究。通过统一的变分波函数我们系统地研究不同维度下极化子到分子的转变,验证这一转变在三维与二维下存在,在一维下不存在。揭示这一转变的本质在于基态从零动量到费米动量的转移,基于此发现了分子态的巨大简并。进一步采用局域密度近似讨论了有限温度有限密度下的单粒子实验可观测量的连续化问题,并与近期实验做了比较。在第四章我们从能谱静态性质的研究来到了非平衡动力学特性研究。利用数值严格对角化讨论了带有纵场的横场伊辛模型从初态为$|Z_2\rangle$出发局域算符期望值的热化相图,选取不同参数区间做局域可观测量的时间演化平均与热力学平均对比来确定是否热化。发现了整个many body scar区域的非热化的行为以及基态附近的弱热化区域,并尝试连接many body scar与非热化严格可解极限。最后第五章,我们总结这一从少体到多体再到动力学的研究。并给出未来值得继续探索问题的思考。

\keywords{杂质,少体,自旋交换,极化子,热化}% 中文关键词


\intobmk\chapter*{Abstract}% 显示在书签但不显示在目录

Cold atom physics has made great progress thanks to the advantage on preparing, manipulating and measuring. Recent attention has not only been focused on intrinsic properties of cold atom systems, but also on quantum simulation for extoic concepts and phenomenas from condensed matter physics community. Many phenomena from solid state physics has been observed on cold atom platforms. More and more models and regions that cannot be explored experimentally in solid state physics have been successfully realized in cold atom. These new experimental progress will show more and more interesting results. Theoretical explanation for these progress and extensions follows quickly. Both of them stimulate each other generously, which makes the whole community flourishing.

This thesis follows along this logic, which focus on recent experimental progress to get down research. In Chapter I we generally introduce some background and methos relating four hot systems, which include mainly: few body system, spin exchange system, fermi polaron and thermalization system. In Chapter II we use exact diagonalization to obtain whole spectrum of two body and three body magnetic impurity system with spin exchange interaction in 1D. We find screening effect in anti-ferromagnetic coupling. We go on illustrate a series of upper branches, in which we highlight one special ferromagnetic branch. This branch has exact wavefunction of spin and charge separation. This few body system can be realized directly and the special correlation coulde be detected. In chapter III we enter into non-magnetic many body impurity system: fermi polaron. We use unified variational wavefunction with up to 2 p-h excitations to systematically study polaron-molecule transition in different dimensions. We confirm the existence of this transition in 3D and 2D. We show that nature of this transition lies on transformation of ground state momentum from 0 to fermi momentum. Furthermore, we show huge degeneracy of molecule states. With the help of local density approximation we calculate trap-average single particle properties in realistic system with finite density and finite temperature. Comparison with experiment data is also shown. In chapter IV we consider non-equilibrium dynamics. By exact diagonalization we study thermalization phase diagram of TFIM with longitudinal field starting from $|Z_2\rangle$. Choosing different parameters, we take comparison between long time average and thermal Gibbs average as justification of thermalization. We found weak thermalization regime, non-thermalization in many body scar regime and try bridging the many body scar and non-thermal integrable limit. Finally, we summarize our research along this logic: from few to many to dynamics and show our thought on direction worthy further investigation in chapter V.

\KEYWORDS{Impurity, Few Body, Spin-exchange, Poalron, Thermalization}% 英文关键词
%---------------------------------------------------------------------------%
