\chapter{总结与展望}\label{chap:summaryandoutlook}

\section{总结}
随着冷原子实验技术的进步,越来越多不同于传统固体物理的实验体系可以在实验中实现。基于这些新的实验体系,有丰富的理论规律以待去探索。本文便沿着样这样一条思路。

我们先从少体杂质体系出发,探究了自旋交换相互作用体系局域自旋杂质与两个巡游费米子相互作用的能谱结构。其中铁磁耦合与反铁磁耦合带来基态特殊的屏蔽效应。而对于激发态upper branch,我们研究了这些特殊的激发态自旋与空间部分行为,利用无限大耦合附近下的简并微扰论研究其能级劈裂行为。其中最为特殊的铁磁upper branch由于对称性的保护与lower branch之间耦合为零,并且具有良好的自旋电荷分离的行为。通过系统地分析少体体系的能谱,我们期望在少体中发现的特殊关联规律可以在多体极限下有所保留。

接着我们从少体研究转到多体物理。我们考虑一个运动杂质与背景费米海相互作用体系的基态转变。采用统一截止到两对电子空穴激发的变分波函数V-2ph,我们理清了在不同维度下的极化子到分子转变的存在性。在三维和二维下,存在这一一阶转变。在二维下我们还用高斯态变分方法进一步证实了这一结论。在一维下我们用V-2ph与贝特假设一起验证了一维下不存在这一转变。 在我们统一波函数的框架下这一转变对应1+N体系基态总动量从零动量到费米动量的转移。我们最终得到结论这一转变来自于粒子空穴激发于泡利不相容原理的共同作用。通过揭示分子态变分波函数为动量为$k_F$的极化子波函数的强耦合渐近描述,我们发现了分子态这一巨大的简并。这一简并在实际体系有限温度有限密度的情况下,会在极化子与分子共存区间增大分子态的占据数。考虑局域密度近似我们计算了实际体系下平均准粒子留数、contact以及极化子能量,并于实验做了细致比较。V-2ph考虑高阶的粒子空穴激发导致准粒子留数降低,因此在弱耦合区间与共振区间与实验符合相比V-1ph较好。

通过上述少体到多体探索,我们研究了杂质体系中相互作用诱导的关联以及维度变化导致的新奇物理。在考虑以上能谱的静态信息之后,我们转向考虑能谱的动力学特征。我们系统地考虑了带有纵场的横场伊辛模型,通过从不同的初态出发,探索了整个参数区间的热化相图。得到不同相区的分界,并对不同相区做了分析。以期对热化动力学有更加系统全面的认知。

\section{展望}
\subsection{少体物理}
在我们计算完成1+2三体体系之后,一个自然的问题是如果继续放入一个巡游费米子变成四体体系会怎样?由于此时系统的自由度距离数值计算的上限自由度还较远,这个问题可以此框架下直接处理。甚至更进一步,我们继续增加巡游费米子,一个自然的问题就是局域杂质束缚巡游费米子的数目是否存在上界?如果存在上界这个上界是由自选所决定吗?以及在反铁磁耦合下增加巡游费米子到近藤物理区间,这一束缚态与近藤单态之间有何联系?这些问题都是值得考虑的。

再者冷原子物理中除了自旋交换相互作用以外,研究较为广泛的还有偶极相互作用,如果在三体体系中将相互作用变成偶极相互作用,那体系的能谱将会有怎样的结构,是否有新的特殊关联出现?这也是值得继续探索的。

\subsection{极化子}
基于我们扩展到有限动量V-2ph,在二维情况下,我们的计算能力以及迭代算法允许我们考虑更高阶的激发V-3ph。在V-3ph中,除了两原子分子态的特殊关联之外,三原子分子态、四原子分子态关联也可以考虑进来。当考虑三、四原子分子态的时候,我们需要进一步放松杂质原子与背景费米子质量比约束。因为这涉及到真空中三体、四体分子态的形成。基于上述简单图像,我们在此方向上有了初步的探索,分别是一个少体体系与多体体系。在少体体系的探索中,我们考虑一个杂质最多可以与多少个背景费米子在真空下形成束缚态。在幺正极限下,我们发现三体分子态出现的临界质量为3.38,四体分子态出新的临界质量为5.14。这两类多体分子态的内禀角动量均为零。但是在动量空间中的粒子空穴关联却展示出很大不同。这种不同可以用原子-分子配对来理解。在多体体系,我们考虑V-3ph变分波函数,研究改变质量比下极化子到三体分子以及四体分子的转变。在超过相应的临界质量比之后,不同极化子到两体分子的一阶转变,极化子到三体分子以及四体分子的转变是一个连续渡越。多体背景下三体与四体关联导致了动量分布晶格化的特征。为进一步探索超越超流配对图像的费米-费米混合气体打开突破口。以上两者扩展都有希望在目前的冷原子实验技术下直接探测验证。


