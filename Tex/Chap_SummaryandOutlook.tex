\chapter{总结与展望}\label{chap:summaryandoutlook}

\section{总结}
随着冷原子实验技术的进步,以原子束缚阱与光晶格作为制备的基础,Feshbach共振和束缚诱导共振等作为调节原子间相互作用的手段,飞行时间、射频谱等作为测量的技术,越来越多固体物理新旧概念可以在冷原子实验中实现。从这些新的实验体系出发,丰富的关联特性有待去探索。本文便沿着样这样一条思路展开研究。

在第一章的绪论中,我们回顾了近期理论与实验结合紧密的几个方向,主要包括少体物理、自旋交换物理、极化子物理以及本征热化物理。我们对相关方向研究做了理论与实验的回顾,熟悉该方向研究背景的,为后续工作的展开做好综述调研。

在第二章中,我们先从少体杂质体系出发,探究了一维下自旋交换相互作用体系局域自旋杂质与两个巡游费米子的少体能谱结构。我们发现了反铁磁耦合给系统基态带来的特殊的屏蔽效应,使得局域自旋杂质只能束缚一个巡游费米子,无法像铁磁耦合那般束缚两个巡游费米子,这是一种特殊的少体关联效应。而对于激发态upper branch,我们研究了这些特殊的激发态波函数中自旋与空间自由度的行为,利用无限大耦合附近下的简并微扰论研究其简并空间劈裂行为。其中最为特殊的铁磁upper branch由于对称性的保护与lower branch之间耦合为零,并且具有良好的自旋电荷分离的行为。通过系统地分析实验少体体系带有的额外接触相互作用,我们研究在实验中观测以上特性的可能性,并简要讨论了少体中发现的特殊关联规律到多体体系的延伸。我们的研究为少体物理严格解家族再添一员,为从少体角度去理解多体物理提供一个新的出发点。

在对少体杂质体系有了较为清楚的物理图像之后,在第三章中我们自然地转向研究多体杂质体系。随着粒子数的增多,杂质与背景原子相互作用,涌现出典型的费米液体行为。我们考虑费米极化子体系,一个运动杂质与背景费米海存在纯接触相互作用。采用统一截止到两对电子空穴激发的变分波函数V-2ph,我们理清了在不同维度下的极化子到分子转变存在性的争议。在三维和二维下,存在这一一阶转变;而在一维下不存在这一转变。在二维下我们还用高斯态变分方法进一步证实这一转变存在,而在一维下我们用V-2ph变分法与贝特假设严格解一起验证这一转变不存在。 在我们统一波函数的框架下这一转变的本质对应1+N体系基态总动量从零动量到费米动量的转移。我们最终得到结论这一转变来自于粒子空穴激发与泡利禁闭机制的共同作用。我们揭示分子态波函数作为动量为$k_F$的极化子波函数在强耦合下的渐近描述,发现分子态存在巨大的简并(三维下为$SO(3)$简并,二维下为$SO(2)简并$)。在考虑实际体系的时候,由于这时存在有限温度和有限密度,这一简并会在极化子与分子共存区间增大分子态的占据数。考虑局域密度近似我们计算了实际体系下平均准粒子留数、Tan接触以及极化子能量,并与近期实验结果做了细致比较。总的来说,V-2ph考虑高阶的粒子空穴激发导致准粒子留数降低,因此在弱耦合区间与共振区间与实验符合相比V-1ph较好。我们的研究厘清了极化子到分子转变在不同维度下的争议,揭示这一转变的本质,并为后续探索极化子中三体、四体关联做好铺垫。

以上的研究聚焦于量子体系的静态能谱性质,我们探究了平衡态物理中由相互作用诱导的关联以及维度变化导致的新奇物理。基于近期里德堡原子实验的启发,在第四章中,我们研究能谱的含时动态性质——非平衡态动力学。我们系统地考虑了带有纵场的横场伊辛模型,通过选取不同的初态,探索了整个参数区间的热化相图。得到不同相区的分界,并对不同相区做了分析。我们的研究有助于热化动力学更系统、全面理解的建立。

综上,我们紧密结合最近冷原子实验进展,沿着从少体到多体再到动力学的思维路线,较为全面地研究了量子系统中新奇的物理特性,以期在理解新奇量子特性研究中更进一步。


\section{展望}
\subsection{少体物理}
在我们计算完成1+2三体体系之后,一个自然的问题是如果继续放入一个巡游费米子变成四体体系会怎样?由于此时系统的自由度距离数值计算的上限自由度还较远,这个问题可以此框架下直接处理。甚至更进一步,我们继续增加巡游费米子,局域杂质束缚巡游费米子的数目是否存在上界?如果存在上界这个上界是由自旋所决定吗?在反铁磁耦合下增加巡游费米子数目到近藤物理区间,这一束缚态与近藤单态之间有何联系?这些问题都是值得进一步探索的。

再者冷原子物理中除了自旋交换相互作用以外,研究较为广泛的还有偶极相互作用,如果在三体体系中将相互作用变成偶极相互作用,那体系的能谱将会有怎样的结构,是否有新的特殊关联出现?这也是值得进一步探究的。

\subsection{极化子}
基于我们扩展到有限动量的变分波函数V-2ph,在二维情况下,我们的计算能力以及迭代算法允许我们考虑更高阶的激发V-3ph。在V-3ph中,除了两原子分子态的特殊关联之外,三原子分子态、四原子分子态关联也可以考虑进来。当考虑三、四原子分子态的时候,我们需要进一步放松杂质原子与背景费米子质量比约束。因为这涉及到真空中三体、四体分子态的形成。基于上述简单图像,我们在此方向上有了初步的探索,分别集中在个少体体系\cite{RuijinUniversal}与多体体系\cite{RuijinEmergence}。在少体体系的探索中,我们考虑一个杂质最多可以与多少个背景费米子在真空下形成束缚态。在幺正极限下,我们发现三体分子态出现的临界质量为3.38,四体分子态出新的临界质量为5.14。这两类多体分子态的内禀角动量均为零。但是在动量空间中的粒子空穴关联却展示出很大不同。这种不同可以用原子-分子配对来理解。在多体体系,我们考虑V-3ph变分波函数,研究改变质量比下极化子到三体分子以及四体分子的转变。在超过相应的临界质量比之后,不同极化子到两体分子的一阶转变,极化子到三体分子以及四体分子的转变是一个连续渡越。多体背景下三体与四体关联导致了动量分布晶格化的特征。为进一步探索超越超流配对图像的费米-费米混合气体打开突破口。以上两者扩展都有希望在目前的冷原子实验技术下直接探测验证。

\subsection{热化动力学}
在本文中,我们主要研究从初态$|Z_2\rangle$出发的动力学。出于普适性的考虑,我们还可以研究$|Z_0\rangle$作为初态的动力学。不同的初态哈密顿量不同参数区间下处于能谱的不同位置,而且整个哈密顿量随着参数的改变,能谱发生不同程度的片段化,由于本征热化假说要求相关能谱密度越密越好,因为这样统计学规律才会发生,因此能谱如何片段化对于系统热化动力学也是有决定性影响的,这些问题仍然是开放的,值得进一步探索。
