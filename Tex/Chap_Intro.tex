\chapter{绪论}\label{chap:kondo}

\section{冷原子概述}


\section{冷原子基本实验技术}

\section{少体严格解}\label{sec:fewbody}
少体严格解引出lower 与upper branch,进而到polaron里面lower与upper branch。
量子三体严格求解。

现代物理学有两个中心观点:还原论与演生论{\color{red}安德森与温伯格}。随着能标不断增加,物质被不断地划分为更小的组成单元,此即还原论。这些更小的组成单元之间通过相互作用的集体协作演生出新的特性,此即演生论。仅有单个组成单元的单体体系是孤立的平凡的,但是两个组成单元加上单元间的相互作用的少体体系则开始出现出非平凡的特性,进一步宏观数目组成单元的多体体系在相互作用影响下涌现出更丰富更新奇的现象。在这个过程中,介于单体与多体之间的的少体体系连接起了微观与宏观的规律。少体物理涵盖很多领域,从宏观世界中两个黑洞的塌缩、天体轨道运行,到微观世界中小分子化合物形成、原子中电子排布、原子核形成、甚至更微观的四夸克态等体系。在这众多体系当中,有一类“人造”少体体系———冷原子少体体系。得益于实验技术的进步,实验中对于微观超冷原子体系的制备、调控和测量进步飞速。这种“人造”少体体系的原子种类数目可控、束缚势场可控、相互作用可调、测量表征手段多样。实验的进步带动了理论研究的复兴。大致来讲,超冷原子少体研究分为Efimov物理与非Efimov物理。

在本章节中我们将有代表性地简要介绍这一领域的相关理论与实验进展。

\subsection{少体理论}
自量子力学诞生初期,少体体系就一直在我们理解自然世界规律中扮演重要角色。诸如谐振子、氢原子等模型的严格解成为我们理解量子力学的出发点。在微观的量子世界中,粒子是全同的,分为费米子与玻色子。对于两体体系(或者单体势场),不论是借助解析手段或者是借助数值手段,我们都可以得到相关体系的精确解。但是随着我们继续增加一个粒子变成三体体系,增加的自由度使得解析求解变得十分困难。这其中的关键在于三体体系已不能用单体或者两体图像来理解,非平凡的关联效应开始出现。更加神奇的是,从微观薛定谔方程过渡到宏观牛顿运动方程,三体体系出现了混沌现象。继续增加一个粒子成为四体体系,解析解则更加难以求得,只能借助数值办法求解。这其中相互作用的形式、外界势场的形状、全同性的影响、不同维度等都带来很高的复杂度。


\subsection{少体实验}

Efimov?

费米化?

有了上面介绍的量子少体研究进展,在一章中将介绍我们的工作,在已有的研究版图中添入新的成员。

\section{自旋交换相互作用}\label{sec:spin-exchange}
spin exchange interaction

\section{极化子理论与实验}


