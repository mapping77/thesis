%---------------------------------------------------------------------------%
%->> Backmatter
%---------------------------------------------------------------------------%
\chapter[致谢]{致\quad 谢}\chaptermark{致\quad 谢}% syntax: \chapter[目录]{标题}\chaptermark{页眉}
%\thispagestyle{noheaderstyle}% 如果需要移除当前页的页眉
%\pagestyle{noheaderstyle}% 如果需要移除整章的页眉
感谢...
\begin{comment}
首先感谢我的指导老师——崔晓玲研究员,感谢崔老师在整个研究期间对我的指导、鼓励和教诲。在做学问、做人、做事各个方面崔老师对我都影响深远。在我刚开始科研的前期,我对于物理的理解浮于表面,对科研浅尝辄止,遇到困难容易产生放弃的心理。崔老师并没有因此而放弃我,而是对我的错误和迷茫给予宽容和开导,辅之以始终如一的一次次耐心的学术讨论,让我一步一步走到今天,来完成我的博士论文。在科研方面,她清晰的物理图像、缜密的逻辑思维、对物理的深刻理解使我受益匪浅,每次在遇到棘手问题不知如何下手的时候,与崔老师的讨论都会及时地使我理清思路,找到切入的方向,一种豁然开朗的感觉如拨云见日。在一次一次的组会讨论中,她对于前沿课题的把握、领域内重要问题的嗅觉指引着我们探索物理的方向,她鼓励我们去问问题,想问题,思考背后的动机,这塑造了我的科研方式。在崔老师身上,我努力去学习新一代年轻物理学家如何做科研,如何做好的科研。在做人做事方面,崔老师脚踏实地,严格要求,毫不拖延的执行力是我要践行去学习的。她对于物理研究的专注与热情时刻激励着我。

感谢清华大学高等研究院的翟荟教授,学习翟老师的冷原子课程与冷原子物理讲义是我理解冷原子物理的基础。翟老师线上线下的科研报告与访谈树立着研究的典范,对于科研品味的追求更是令人心驰神往。感谢武汉物理数学研究所的管习文研究员,与管老师关于极化子到分子转变中分数统计的讨厘清了我的疑惑,管老师的报告以及个人研究经历的分享使我受益良多,从澳大利亚“背”回问题的精神值得我努力学习。 感谢中国科学技术大学的易为教授,短暂地几次与易为老师关于非厄米物理的交流开阔了我的研究视野,易为老师的鼓励激励着我前行。

感谢麻银峰师兄和刘瑞金师兄,我们的合作使我受益良多,遇到难点我们互相讨论解答,互相启发。感谢周黎红师姐、江慧师姐、潘磊师兄、王健师兄、梁辰光师兄、给予我的帮助与指导,不断解答我的疑惑。感谢张越水、张帅、刘春晖、张华琛等同学,与你们的讨论与交流使我了解到更多领域的研究。

感谢齐建为老师,感谢您的教导与辛勤工作,为我们的学习、生活提供了很多便利条件。感谢管理部门各位老师的辛勤付出。

感谢彭雪琦同学一路走在来的鼓励和陪伴,希望我们会有一个美好的未来!

最后感谢我的父母家人,感谢他们给予我的理解与支持,感谢他们的关心与照顾,



\end{comment}

\chapter{作者简历及攻读学位期间发表的学术论文与研究成果}


\section*{作者简历:}

彭程,山东省潍坊市人,中国科学院物理研究所博士研究生。

\section*{已发表(或正式接受)的学术论文:}

{
\setlist[enumerate]{}% restore default behavior

\begin{enumerate}[nosep]
    \item {\bfseries\sffamily Cheng Peng}, Xiaoling Cui, Few-body solutions under spin-exchange interaction: Magnetic bound state and the Kondo screening effect, Phys. Rev. A. 102, 033312(2020)
    
    \item {\bfseries\sffamily Cheng, Peng}*, Ruijin Liu*, Wei Zhang , Xiaoling Cui, Nature of the polaron-molecule transition in Fermi polarons, Phys. Rev. A. 103, 063312(2021)

    \item Yinfeng Ma, {\bfseries\sffamily Cheng Peng}, Xiaoling Cui, Borromean Droplet in Three-Component Ultracold Bose Gases, Phys. Rev. Lett. 127, 043002(2021)

    \item Ruijin Liu, {\bfseries\sffamily Cheng Peng}, Xiaoling Cui, Universal tetramer and pentamer in two-dimensional fermionic mixtures, arXiv:2202.01437

    \item Ruijin Liu, {\bfseries\sffamily Cheng Peng}, Xiaoling Cui, Emergence of Crystalline Few-body Correlations in Mass-imbalanced Fermi Polarons, arXiv:2202.03623

    \item {\bfseries\sffamily Cheng Peng}, Xiaoling Cui, Thermalization phase diagram in transverse field Ising  model with longitudinal field, under preaparation.
\end{enumerate}
}


\section*{未发表的学术论文:}

{
\setlist[enumerate]{}% restore default behavior

\begin{enumerate}[nosep]
    \item {\bfseries\sffamily Cheng Peng}, Xiaoling Cui, Bridging quantum many body scars and integrable non-thermal limit, under preaparation.
\end{enumerate}
}



\section*{参加的研究项目及获奖情况:}
...

\cleardoublepage[plain]% 让文档总是结束于偶数页,可根据需要设定页眉页脚样式,如 [noheaderstyle]
%---------------------------------------------------------------------------%
