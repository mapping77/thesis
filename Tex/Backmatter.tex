%---------------------------------------------------------------------------%
%->> Backmatter
%---------------------------------------------------------------------------%
\chapter[致谢]{致\quad 谢}\chaptermark{致\quad 谢}% syntax: \chapter[目录]{标题}\chaptermark{页眉}
%\thispagestyle{noheaderstyle}% 如果需要移除当前页的页眉
%\pagestyle{noheaderstyle}% 如果需要移除整章的页眉

首先感谢我的指导老师——崔晓玲研究员,感谢崔老师在整个研究期间对我的指导、鼓励和教诲。在做学问、做人、做事各个方面崔老师对我都影响深远。在我刚开始科研的前期,我对于物理的理解浮于表面,对科研浅尝辄止,遇到困难容易产生放弃的心理。崔老师并没有因此而放弃我,而是对我的错误和迷茫给予宽容和开导,辅之以始终如一的一次次耐心的学术讨论,让我一步一步走到今天,来完成我的博士论文。在科研方面,她清晰的物理图像、缜密的逻辑思维、对物理的深刻理解使我受益匪浅,每次在遇到棘手问题不知如何前进的时候,与崔老师的讨论都会及时地使我理清思路,找到切入的方向,一种豁然开朗的感觉如拨云见日。在一次一次的组会讨论中,她对于前沿课题的把握、领域内重要问题的嗅觉指引着我们探索物理的方向,她鼓励我们去问问题,想问题,思考背后的动机,这塑造了我的科研方式。在崔老师身上,我努力去学习新一代年轻物理学家如何做科研,如何做好的科研。在做人做事方面,崔老师脚踏实地,严格要求,毫不拖延的执行力是我要践行去学习的,她对于物理研究的专注与热情时刻激励着我。

感谢清华大学高等研究院的翟荟教授,翟老师的冷原子课程与《Ultracold Atomic Physics》是我理解冷原子物理的基础。翟老师线上线下的科研报告与访谈树立着研究的典范,对于科研品味的追求更是令人心驰神往,感谢翟老师关于动力学研究的建议。感谢武汉物理数学研究所的管习文研究员,与管老师关于极化子到分子转变中分数统计的讨论厘清了我的疑惑,管老师的报告以及个人研究经历的分享使我从中学到很多,从澳大利亚“背”回问题的精神值得我努力学习。 感谢中国科学技术大学的易为教授,短暂地几次与易为老师关于非厄米物理的交流开阔了我的研究视野,感谢易为老师的鼓励。感谢人民大学张威老师一起合作的极化子相关研究,感谢他对论文提出的建议与修改指导。感谢所内的陈澍研究员,陈老师科研报告以及对研究组内同学的关怀对我帮助很大。

感谢Rutgers大学的Natan Andrei教授关于近藤物理严格解细节的回信解答,时隔接近三十年的细节交流展现了他一丝不苟、严谨负责的科学研究之风,使我肃然起敬。

感谢王建师兄、麻银峰师兄和刘瑞金师兄,我们的合作使我受益良多,我们互相讨论、互相启发。感谢潘磊师兄耐心细致的交流,感谢他关于少体研究、非厄米研究、动力学研究的讨论与建议。感谢周黎红师姐、刘彦霞师姐、姜慧师姐、梁辰光师兄给予我的帮助与指导,不断解答我的疑惑。感谢张越水、张帅、刘春晖、张华琛、高顺业、杨辉、乔木、段昊武、赵致远、张志城等同学,与你们的讨论与交流不断巩固我对于物理的理解,并且使我了解到了更多凝聚态物理领域的研究。感谢陆建强同学关于计算机、数学、物理的分享及交流,这开阔了我的视野。

感谢齐建为老师,感谢她五年来的教导与辛勤工作,为我们的学习、生活提供了诸多便利。感谢管理部门各位老师的辛勤付出。

感谢彭雪琦同学一路走来的鼓励和陪伴,感谢她在我迷茫的时候的鼓励,在我轻浮时候的忠告,在我失落时候的开导,她的出现为生活增添了新的色彩。

感谢我的父母,感谢父母的养育之恩,感谢家人,他们从小尊重我的选择,给予我信任、理解与支持。

最后感谢这段学习与研究的经历,这期间学到的研究能力是求学以来最大的收获,我将继续保持并践行下去!

\chapter{作者简历及攻读学位期间发表的学术论文与研究成果}


\section*{作者简历}

彭程,山东省潍坊市人,中国科学院物理研究所博士研究生。

通讯地址:北京市海淀区中关村南三街8号M楼 

邮编:100190

E-mail: pcheng77@126.com

\section*{教育背景}

2017.09 - 2022.06 中国科学院物理研究所 理学博士

2013.09 - 2017.06 山东大学 理学学士

\section*{已发表(或正式接受)的学术论文}

{
\setlist[enumerate]{}% restore default behavior

\begin{enumerate}[nosep]
    \item {\bfseries\sffamily Cheng Peng}, Xiaoling Cui, Few-body solutions under spin-exchange interaction: magnetic bound state and the Kondo screening effect, Phys. Rev. A. 102, 033312(2020)
    
    \item {\bfseries\sffamily Cheng Peng}*, Ruijin Liu*, Wei Zhang, Xiaoling Cui, Nature of the polaron-molecule transition in Fermi polarons, Phys. Rev. A. 103, 063312(2021)

    \item Yinfeng Ma, {\bfseries\sffamily Cheng Peng}, Xiaoling Cui, Borromean droplet in three-component ultracold bose gases, Phys. Rev. Lett. 127, 043002(2021)

    \item Ruijin Liu, {\bfseries\sffamily Cheng Peng}, Xiaoling Cui, Universal tetramer and pentamer in two-dimensional fermionic mixtures, arXiv:2202.01437

    \item Ruijin Liu, {\bfseries\sffamily Cheng Peng}, Xiaoling Cui, Emergence of crystalline few-body correlations in mass-imbalanced Fermi polarons, arXiv:2202.03623
\end{enumerate}
}


\section*{未发表的学术论文}

{
\setlist[enumerate]{}% restore default behavior

\begin{enumerate}[nosep]
    \item {\bfseries\sffamily Cheng Peng}, Xiaoling Cui, Bridging quantum many body scars and integrable non-thermal limit, under preaparation.
\end{enumerate}
}



\section*{参加学术报告}

2022.02 费米极化子研讨会:从冷原子到二维半导体

2021.07 第十四届冷原子物理青年学者学术讨论会 (海报:Nature of polaron-molecule transition in Fermi polarons)

2020.11 量子自束缚液滴研讨会

2018.04 武汉冷原子中的少体和多体物理


\section*{获奖情况}

2021.12 物理所所长奖学金表彰奖

2021.07 第十四届冷原子物理青年学者学术讨论会博士生论坛优秀学术交流奖

2020.12 物理所所长奖学金表彰奖

\cleardoublepage[plain]% 让文档总是结束于偶数页,可根据需要设定页眉页脚样式,如 [noheaderstyle]
%---------------------------------------------------------------------------%
