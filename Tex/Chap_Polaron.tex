\chapter{极化子-分子转变的本质}\label{chap2polaron}

在论文的介绍部分{\color {red} 第一章}我们介绍了费米极化子的理论与实验研究脉络。其中三维和二维下极存在化子到分子的转变。一直以来这一转变的证据来自于两个分立的变分波函数。在我们最新的工作中,我们用一个截止到2对粒子-空穴对激发的统一变分波函数V-2ph来研究极化子-分子的转变。我们证实在三维和二维下确实存在极化子到分子的一阶转变。通过这一统一的变分波函数我们给出这一转变的本质在于不同总动量$\Vector{Q}=0$和$|\Vector{Q}|=k_F$和的转变。这里的$\Vector{Q}$指的是总动量相对于费米海的动量。之前研究中广泛使用的分子态变分波函数是我们V-2ph变分波函数$|\Vector{Q}|=k_F$在强相互作用下的渐近极限,并因此带来很大的$SO(3)$(三维是$SO(3)$,二维是$SO(2)$)基态简并。这一简并的发现导致了态密度的变化,这一变化改变了有限温度和有限密度下分子态的占据。我们与实验观测到的结果相比较,可以在弱耦合和共振区间得到较好的符合。我们进一步在二维下采用了高斯态的变分办法,验证了我们提出的极化子到分子的转变的本质在于动量的转变。在一维下变分法给出没有一阶转变,与贝特假设严格解的结果一致。最终得到极化子-分子的转变存在性由费米统计原理和维度带来的量子涨落共同决定。

\section{引言}
在论文的开始部分我们梳理了极化子理论与实验研究的进展。其中我们提到对于高维的情况,随着杂质原子与背景费米海之间相互作用的增强,系统基态发生极化子到分子的转变。相应的物理图像也很直观,当相互作用不那么强的时候杂质的运动被背景费米海的粒子空穴对激发所修饰,单粒子性质被重整化。一旦相互作用越过临界点,一个杂质可以与背景费米海的一个原子结合在一起形成服从玻色统计的分子态,不过这个分子态是有费米海存在的分子态,并不是真空的分子态。通常用两个独立的变分波函数来表征:
\begin{equation}
\begin{aligned}
&P_{2 n+1}(0) \\
&\quad=\left[\psi_{0} c_{\mathbf{0}_{\downarrow}}^{\dagger}+\sum_{l=1}^{n} \sum_{\mathbf{k}_{i} \mathbf{q}_{j}} \psi_{\mathbf{k}_{\mathbf{q}_{j}}} c_{\mathbf{P} \downarrow}^{\dagger} \prod_{i=1}^{l} c_{\mathbf{k}_{i} \uparrow}^{\dagger} \prod_{j=1}^{l} c_{\mathbf{q}_{j} \uparrow}\right]|\mathrm{FS}\rangle_{N}\\
M_{2 n+2}(0)=& {\left[\sum_{\mathbf{k}} \phi_{\mathbf{k}} c_{-\mathbf{k}, \downarrow}^{\dagger} c_{\mathbf{k}, \uparrow}^{\dagger}+\sum_{l=1}^{n} \sum_{\mathbf{k}_{i} \mathbf{q}_{j}} \phi_{\mathbf{k}_{i} \mathbf{q}_{j}} c_{\mathbf{P} \downarrow}^{\dagger}\right.} \\
&\left.\times \prod_{i=1}^{l+1} c_{\mathbf{k}_{i} \uparrow}^{\dagger} \prod_{i=1}^{l} c_{\mathbf{q}_{j} \uparrow}\right]|\mathrm{FS}\rangle_{N-1}
\end{aligned}
\end{equation}
这里$\hat{c}^\dagger_{\Vector{k},\sigma}$为动量为$\Vector{k}$的自旋$1/2$费米子。在我们接下来中取自旋向上为背景原子,$|FS\range_N$为N个原子的费米海。自旋向下为杂质原子。求和中的$\Vector{q}$限制在费米球内部,$\Vector{k}$限制在费米海外部。其中$\Vector{P}=\sum_{j} \Vector{q}_{j}-\sum_{i} \Vector{k}_{i}$,通过分别将变分波函数带入到薛定谔方程中可以得到极化子分子一阶转变的图像。尽管这种分立的变分波函数物理上很直观,但是这种两个预先选定的变分波函数来给出相变带有很大的认为选择因素。对此最直接的一个不完善点在于一个观察\cite{edwards2013smooth}:该研究发现以上两个变分波函数通过考虑不同的角动量和不同阶的粒子空穴对激发下是互相包含的。因此在这个意义下,这种分立的变分波函数需要小心对待。

而在实验这边,实验解析到的准粒子留数是连续地趋向于零,并没有出现跳变。尤其是最近的实验通过动量分辨的Raman谱直接解析到不同物理量在转变前后连续的变化\cite{Sagi2020}。再一次印证了变分波函数下极化子到分子的转变需要谨慎对待。

基于上面理论与实验方面的动机,在最近的研究中,通过统一的变分波函数来研究极化子与分子的必要性被重视起来\cite{Cui2020Fermi}。通过统一的变分波函数$V-1ph$:$P_3(\Vector{Q}),即将动量扩展到非零动量$来研究三维极化子。我们这里的$\Vector{Q}$指的是系统总的动量减掉费米海$|FS\range_N$的动量。

\section{模型}

\section{结果}

\section{小结与展望}

